\documentclass[a4paper, fontsize=14pt]{article}
\usepackage{style}
\setcounter{page}{0} %в зависимости от того, какой по счёту страницей должно быть оглавление!

\begin{document}

\tableofcontents

\newpage

\section*{Введение}
\addcontentsline{toc}{section}{Введение}
\newpage
% ...


\section*{Уравнение распространения волн в сейсмической среде с затуханием} 
\addcontentsline{toc}{section}{Уравнение распространения волн в сейсмической среде с затуханием}

На практике сейсмические среды с эффектом затухания описываются линейной теорией вязкоупругости.
Эта теория учитывает как упругие, так и вязкие свойства материалов,
позволяя более точно моделировать поведение материалов под воздействием сейсмических нагрузок. 
Подробную информацию можно найти в книгах \cite{Christenses,Mainardi,Carcione,Borcherdt}, здесь же ограничимся
приведением основных формул и математических формулировок в контексте задачи распространения волн.

Будем рассматривать инфинитезимальные напряжения ($\varepsilon_{ij}$) и малые перемещения ($u_{ij}$).
В этом случае они будут связаны уравнением:

\begin{equation}
    \label{eq:strain_u_rel}
    \varepsilon_{ij} = \frac{1}{2} \left( \frac{\partial u_i}{\partial x_j} + \frac{\partial u_j}{\partial x_i}\, \right). 
\end{equation}

Также будем считать, что история деформаций ($\varepsilon_{ij}(t)$) предполагается непрерывной, а функционал преобразующий историю изменения деформаций ($\varepsilon_{ij}(t)$) в соответствующую историю изменения напряжений ($\sigma_{ij}(t)$) предполагается линейным (подробнее см. \cite{Christenses})

Тогда в самом общем виде этот функционал, который будем называть определяющим уравнением, может быть записывается следующим образом:

\begin{equation}
    \label{eq:hooks_law}
    \sigma_{ij}(t) = \int_{-\infty}^{\infty} G_{ijkl} (t - s) \frac{\partial \varepsilon_{kl} (s)}{\partial s} d s = G_{ijkl}(t) \ast  \frac{\partial \varepsilon_{kl} (t)}{\partial t} ,
\end{equation}
где $G_{ijkl}(t)$ нызывается релаксационным тензором, причем компоненты тензора $G_{ijkl}$ равны нулю при $t < 0$.

Такая запись определяющего соотношения позволяет увидеть, что оно удовлетворяет двум фундаментальным гипотезам теории линейной вязкоупругости:
\begin{enumerate}
    \item инвариантность по отношению к переносу по времени;
    \item принцип причинности (отклик системы на возмущение зависит только от предыдущих событий во времени).
\end{enumerate}

Поскольку $\sigma_{ij}$ и $\varepsilon_{ij}$ являются симметричными тензорами, то $G_{ijkl}$ также симметричный тензор: $G_{ijkl} = G_{jikl} = G_{ijlk}$.


Для изотропной среды функция релаксация имеет вид:
\begin{equation}
    \label{eq:relax_function}
    G_{ijkl} (t) = \frac{1}{3} \left( G_{b}(t) - G_{s}(t) \right) \delta_{ij} \delta_{kl} + \frac{1}{2} G_s \left( \delta_{ik} \delta_{jl} + \delta_{il} \delta_{jk}\right), 
\end{equation}
где $G_s$ и $G_b$ пространственно независимые релаксационные функции, характеризующие сдвиговые и объемные свойства материала при соответствующих деформациях; 
$\delta_{ij}$ --~  символ Кронекера;

Часто уравнение \eqref{eq:relax_function} переписывают в терминах, аналогичных константам Ламе из классической теории упругости. Для этого можно сделать замену:
\begin{equation}
    \label{eq:lame_mu}
    \mu(t) = \frac{G_s(t)}{2},
\end{equation}
\begin{equation}
    \label{eq:lame_k}
    k(t) = \frac{G_b(t)}{3},
\end{equation}
\begin{equation}
    \label{eq:lame_lambda}
    \lambda(t) = \frac{G_k(t) - G_s(t)}{3}  = k(t) - \frac{2}{3} \mu(t).
\end{equation}

Тогда подставив \eqref{eq:lame_mu}, \eqref{eq:lame_k}, \eqref{eq:lame_lambda} в соотношение \eqref{eq:relax_function} получим:

\begin{equation}
    \label{eq:relax_tensor_lame}
    G_{ijkl} = \lambda \delta_{ij} \delta_{kl} + \mu \left( \delta_{ik} \delta_{jl} + \delta_{il} \delta_{jk}\right)
\end{equation}

В нотации Фойгта тензор релаксации \eqref{eq:relax_tensor_lame} будет иметь такой же вид, как и в классической теории упругости:

\begin{equation*}
    G = \left(\begin{array}{ccc|ccc}
        \lambda + 2 \mu & \lambda & \lambda & 0 & 0 & 0 \\
        \lambda & \lambda + 2 \mu & \lambda & 0 & 0 & 0 \\
        \lambda & \lambda & \lambda + 2 \mu & 0 & 0 & 0 \\
        \hline
        0 & 0 & 0 & \mu & 0 & 0 \\
        0 & 0 & 0 & 0 & \mu & 0 \\
        0 & 0 & 0 & 0 & 0 & \mu 
        \end{array}\right)\,
\end{equation*}

Подставим \eqref{eq:relax_tensor_lame} в \eqref{eq:hooks_law} получим следующее выражение, похожее на упругий закон Гука:
\begin{equation}
    \label{eq:hooks_law_lame}
    \sigma_{ij} = \delta_{ij} \; \lambda \ast \frac{\partial \varepsilon_{kk}}{\partial t} + 2 \mu \ast \frac{\partial \varepsilon_{ij}}{\partial t}
\end{equation}

Для модели стандартного линейного тела (см. \cite{Mainardi,Carcione}), используя $\tau$-метод (см. \cite{Bohlen}), выводятся функции $\mu(t)$ и $k(t)$ в виде:

\begin{equation}
    \label{eq:mu}
    \mu(t) = M_s \left[ 1 - \left(1 - \frac{\tau_s}{\tau}\right) \exp \left( - \frac{t}{\tau} \right)\right] H (t) = \hat{\mu}(t) \; H(t),
\end{equation}

\begin{equation}
    \label{eq:k}
    k(t) = M_b \left[ 1 - \left(1 - \frac{\tau_b}{\tau}\right) \exp \left( - \frac{t}{\tau} \right)\right] H (t) = \hat{k}(t) \; H(t),
\end{equation}
где $H(t)$ функция Хэвисайда.

    Также введем следущее обозначение:
    \begin{equation}
        \hat{\lambda}(t) = \hat{k}(t) - \frac{2}{3}\hat{\mu}(t)
    \end{equation}

Тогда уравнение \eqref{eq:hooks_law_lame} можно переписать в виде:
\begin{equation}
    \label{eq:hooks_law_hat}
    \sigma_{ij}(t) = \delta_{ij} \int_0^t \hat{\lambda}(t - s) \; \frac{\partial \varepsilon_{kk}(s)}{\partial s} ds + 2 \int_0^t \hat{\mu}(t - s) \; \frac{\partial \varepsilon_{ij}(s)}{\partial s} ds  
\end{equation}

Теперь выведем дифференциальное уравнение движения в виде системы дифференциальных уравнений в частных производных 1-го порядка.

Проинтегрируем по частям \eqref{eq:hooks_law_hat}:

\begin{equation}
    \label{eq:hooks_law_int_by_part}
    \begin{aligned}
        \sigma_{ij}(t) = \; \delta_{ij} &\left\{\hat{\lambda}(0) \; \varepsilon_{kk}(t) + \int_0^t \frac{\partial \hat{\lambda}(t - s)}{\partial s}  \varepsilon_{kk}(s) ds \right\} + \\
        + & 2 \left\{\hat{\mu}(0) \; \varepsilon_{ij}(t) + \int_0^t \frac{\partial \hat{\mu}(t - s)}{\partial s}  \varepsilon_{ij}(s) ds \right\}
    \end{aligned}
\end{equation}

Продифференцируем \eqref{eq:hooks_law_int_by_part} по $t$:
\begin{equation}
    \label{eq:hooks_law_int_by_part_differentiate}
    \begin{aligned}
        \frac{\partial \sigma_{ij}}{\partial t}  = \; \delta_{ij} &\left\{\hat{\lambda}(0) \; \frac{\partial \varepsilon_{kk}}{\partial t}  + \frac{d}{dt} \int_0^t \frac{\partial \hat{\lambda}(t - s)}{\partial s}  \varepsilon_{kk}(s) ds \right\} + \\
        + & 2 \left\{\hat{\mu}(0) \; \frac{\partial \varepsilon_{ij}}{\partial t} + \frac{d}{dt} \int_0^t \frac{\partial \hat{\mu}(t - s)}{\partial s}  \varepsilon_{ij}(s) ds \; \right\}
    \end{aligned}
\end{equation}

Заметим, что:
\begin{equation}
    \label{eq:mu_t}
    \frac{\partial \hat{\mu}}{\partial t} = \frac{M_s}{\tau} \left[1 - \frac{\tau_s}{\tau} \right] \exp \left( - \frac{t}{\tau}\right), \; \text{причем} \;  \frac{\partial^n \hat{\mu}}{\partial^n t} = (-1)^{n - 1} \frac{1}{\tau^{n-1}} \frac{\partial \hat{\mu}}{\partial t}, \; n \geq 2. 
\end{equation}

Аналогично,
\begin{equation}
    \label{eq:k_t}
    \frac{\partial \hat{k}}{\partial t} = \frac{M_b}{\tau} \left[1 - \frac{\tau_b}{\tau} \right] \exp \left( - \frac{t}{\tau}\right)  \; \text{и} \;  \frac{\partial^n \hat{k}}{\partial^n t} = (-1)^{n - 1} \frac{1}{\tau^{n-1}} \frac{\partial \hat{k}}{\partial t}, \; n \geq 2. 
\end{equation}



Введем новые переменные:
\begin{equation*}
    \begin{aligned}
        r_{ij} &= \delta_{ij} \; \frac{d}{dt} \int_0^t \frac{\partial \hat{\lambda}(t - s)}{\partial s}  \varepsilon_{kk}(s) ds + 2 \frac{d}{dt} \int_0^t \frac{\partial \hat{\mu}(t - s)}{\partial s}  \varepsilon_{ij}(s) ds = \\
           & = \delta_{ij} \left[\; \frac{d \hat{\lambda}}{d t}\Big{|}_{t = 0} \;  \; \varepsilon_{kk} (t) + \int_0^t \frac{\partial^2 \hat{\lambda}(t - s)}{\partial^2 s} \varepsilon_{kk}(s) ds \right] + \\ 
           & + 2 \left[ \frac{d \hat{\mu}}{d t}\Big{|}_{t = 0} \;  \varepsilon_{ij} + \int_0^t \frac{\partial^2 \hat{\mu}(t - s)}{\partial^2 s} \varepsilon_{ij}(s) ds \right] 
    .\end{aligned}
\end{equation*}

Используя \eqref{eq:mu_t} и \eqref{eq:k_t} можно записать:

\begin{equation}
    \label{eq:r_rare}
    \begin{aligned}
        r_{ij} &= \delta_{ij} \left[\; \frac{d \hat{\lambda}}{d t}\Big{|}_{t = 0} \;  \; \varepsilon_{kk} (t) - \frac{1}{\tau} \int_0^t \frac{\partial \hat{\lambda}(t - s)}{\partial s} \varepsilon_{kk}(s) ds \right] + \\ 
           & + 2 \left[ \frac{d \hat{\mu}}{d t}\Big{|}_{t = 0} \;  \varepsilon_{ij} - \frac{1}{\tau} \int_0^t \frac{\partial \hat{\mu}(t - s)}{\partial s} \varepsilon_{ij}(s) ds \right].
    \end{aligned}
\end{equation}

Продифференцируем \eqref{eq:r_rare} по времени:

\begin{equation*}
    \begin{aligned}
        \frac{d r_{ij}}{dt} &= \delta_{ij} \left[\; \frac{d \hat{\lambda}}{d t}\Big{|}_{t = 0} \;  \; \frac{d \varepsilon_{kk}}{dt}  (t) - \frac{1}{\tau} \frac{d}{dt} \int_0^t \frac{\partial \hat{\lambda}(t - s)}{\partial s} \varepsilon_{kk}(s) ds \right] + \\ 
           & + 2 \left[\frac{d \hat{\mu}}{d t}\Big{|}_{t = 0} \; \frac{d \varepsilon_{ij}}{dt}  - \frac{1}{\tau} \frac{d}{dt} \int_0^t \frac{\partial \hat{\mu}(t - s)}{\partial s} \varepsilon_{ij}(s) ds \right].
    \end{aligned}
\end{equation*}

Теперь имеем два случая:
\begin{equation}
    \label{eq:brain_funct}
    \begin{aligned}
        1.& \; i = j \Rightarrow \frac{dr_{ii}}{dt} =    \left(\frac{M_b}{\tau} \left[1 - \frac{\tau_b}{\tau} \right] - \frac{2}{3} M_s \left[1 - \frac{\tau_s}{\tau}\right] \right) \frac{d \varepsilon_{kk}}{dt} + \\
        &\qquad +\frac{2 M_s}{\tau} \left[1 - \frac{\tau_s}{\tau} \right] \frac{d \varepsilon_{ii}}{dt} - \frac{1}{\tau} r_{ii} \\
        2.& \; i \neq j \Rightarrow \frac{dr_{ij}}{dt} = \frac{2 M_s}{\tau} \left[1 - \frac{\tau_s}{\tau} \right] \frac{d \varepsilon_{ij}}{dt} - \frac{1}{\tau} r_{ij} 
    \end{aligned}
\end{equation}

Переменные $r_{ij}$ отвечают за историю процесса, поэтому называются функциями памяти. Если бы они равнялись нулю, то мы бы получили классическое формулы из теории упругости.

Наконец, введя функции памяти, уравнение \eqref{eq:hooks_law_int_by_part_differentiate} будет иметь вид:
\begin{equation}
    \label{eq:hooks_law_final}
    \begin{aligned}
        \frac{\partial \sigma_{ij}}{\partial t}  = \; \delta_{ij} \left( M_b\frac{\tau_b}{\tau} - \frac{2}{3} M_s \frac{\tau_s}{\tau} \right)  \frac{\partial \varepsilon_{kk}}{\partial t} + 2 M_s \frac{\tau_s}{\tau} \frac{\partial \varepsilon_{ij}}{\partial t} + r_{ij}
    \end{aligned}
\end{equation}

Запишем уравнение баланса импульса для сплошной среды: 
\begin{equation}
    \label{eq:newton_law}
    \rho \frac{\partial^2 u_{i}}{\partial^2 t} = \frac{\partial \sigma_{ij}}{\partial x_j}  + f_{i} \quad \text{или} \quad \rho \frac{\partial v_{i}}{\partial t}  = \frac{\partial \sigma_{ij}}{\partial x_j}  + f_{i},
\end{equation}
где $u_{i}, v_{i}$ компоненты векторов перемещения и скорости перемещения, $\sigma_{ij}$ компонент симметричного тензора напряжения, $f_i$ компонент вектора внешних массовых сил, $\rho$ плотность среды.

Таким образом, подстав в \eqref{eq:brain_funct}, \eqref{eq:hooks_law_final} продифференцированное по времени соотношение \eqref{eq:strain_u_rel}, будем иметь замкнутую систему дифференциальных уравнений \eqref{eq:brain_funct}, \eqref{eq:hooks_law_final} и \eqref{eq:newton_law}, описывающую движение в вязкоупругой среде.



\newpage
\section*{Заключение}
\addcontentsline{toc}{section}{Заключение}

\newpage

\addcontentsline{toc}{section}{Список литературы}

\begin{thebibliography}{9}
    \addcontentsline{toc}{section}{\refname}
    \bibitem{Christenses} Введение в теорию вязкоупругости / Р. Кристенсен ; пер. с англ. М. И. Рейтмана ; под ред. Г. С. Шапиро, 1974. - 338 с
    \bibitem{Mainardi} Fractional Calculus and Waves in Linear Viscoelasticity
    \bibitem{Carcione} Wave in real solid
    \bibitem{Borcherdt} Viscoelastic waves and rays in layered media
    \bibitem{Bohlen} T. Bohlen, “Parallel 3-D viscoelastic finite difference seismic modelling,” Comput. Geosci. 28, 887–899
    (2002).
\end{thebibliography}

\end{document}